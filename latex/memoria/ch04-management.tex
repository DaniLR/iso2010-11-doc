\chapter{Procesos de gestión}

En este capítulo se muestra una descripción no técnica de los procesos que han
influido en el desarrollo de la práctica.

\section{Actividades}

Para el desarrollo de esta práctica, se han desempeñado las siguientes
actividades.

\subsection*{Especificación de requisitos}

Realizada durante las primeras dos semanas, esta actividad consitió en una
lectura detallada del enunciado de la práctica para extraer las funcionalidades
que la aplicación debía tener.

Del desarrollo de esta actividad se obtuvo el documento \textit{UC 01-002.01
Especificación de requisitos software}, que describe de una manera más
detallada y directa el alcance de la aplicación. Además se generó el diagrama
de modelos de casos de uso, que muestra esquemáticamente todos los requisitos
funcionales de la aplicación.

\subsection*{Análisis y diseño}

Partiendo de la especificación de requisitos, los analistas-diseñadores
realizarán en esta actividad una serie de acciones que definirán el diseño de
la arquitectura de la aplicación y las relaciones existentes entre las clases
de la misma.

Resultado de esta actividad habrá como mínimo un \textit{diagrama de secuencia}
por cada caso de uso. En estos diagramas se muestra en un nivel de detalle medio
el flujo de mensajes entre clases para la consecución de un determinado caso de
uso.

Con el conocimiento obtenido del análisis de cada caso de uso se irán creando de
manera incremental los \textit{diagramas de clases}. El resultado son tres
diagramas por cada capa arquitectónica, y un cuarto diagrama de clases que
muestra las relaciones entre las clases más importantes a alto nivel.

\subsection*{Implementación}

Esta actividad materializa el diseño realizado en la actividad anterior. El
resultado será una nueva versión de la aplicación con la funcionalidad
correspondiente.

La brevedad de este apartado no refleja el tiempo invertido en esta actividad
con respecto al resto de actividades. Esta actividad no se limita a programar,
sino que a veces es necesario resolver problemas técnicos que pueden sergir
durante la programación.

\subsection*{Pruebas}

Con esta actividad se busca la validación y verificación de los componentes de
la aplicación, a nivel tanto individual como de conjunto. Las pruebas
se han realizado usando el \textit{framework} de pruebas \textit{JUnit} en
combinación con el analizador de covertura \textit{EclEmma}.

De esta actividad se ha obtenido una extensa batería de pruebas que prueban el
comportamiento a nivel de integración de toda la capa de dominio.

Por su excesiva complejidad, se decidió no probar programáticamente la capa de
presentación de la aplicación. En su lugar, se crearían guiones de pruebas
exploratorias.

\subsection*{Documentación}

Actividad de cierre del proyecto, dentro de esta actividad se incluye la
redacción del presente documento, los informes de las pruebas realizadas en la
actividad anterior, o el manual de usuario de la aplicación.

En el entregable final, presentado en un DVD, se incluirán todos los artefactos
generados por las actividades aquí explicadas.

\section{Planificación}

De la planificación mostrada en el documento \textit{UC 01-001.02 Plan de
gestión de proyecto software} se podía decir en su día que era muy optimista. A
día de hoy, se puede decir que aquella planificación era extremadamente
optimista.

Únicamente las dos o tres primeras semanas se cumplieron hasta cierto grado.
Sin embargo la incertidumbre de no estar aún disponible la interfaz de
comunicaciones ralentizó el proceso de analisis y diseño hasta pararlo por
completo.

Se tomó la decisión de no continuar con el desarrolló hasta que la interfaz no
estuviera definida para no realizar trabajo que tuviera que ser desechado más
adelante por no ajustarse a la especificación.

Aún con todo, retrospectivamente se pueden apreciar distintas fases en el
desarrollo que en mayor o menor medida conciden con las del proceso unificado.

\subsection{Fase de inicio}

\textit{Del 6 de octubre al 31 de octubre}

Durante estas aproximadamente tres semanas y media, las actividades realizadas
se ajustaron a la planificación expuesta en el plan de proyecto. En este
período se redactaron el plan gestión de proyecto software y la especificación
de requisitos software.

En los últimos días, se inicio con las actividades de análisis y diseño. Se
llegaron a cubrir los primeros casos de uso, todos relacionados con la gestión
de usuarios.

\subsection{Fase de elaboración}

\textit{Del 1 de noviembre al 25 de diciembre}

Esta fase estuvo marcada por la falta de una interfaz de comunicaciones
definida y estable. Se necesitaba avanzar en el proceso de diseño, pero a la
hora de definir las clases, sus atributos, y los argumentos usados en los
métodos, siempre surgía la duda de qué datos se usarían.

Esto no significa que el equipo estuviera totalmente parado, pero sí que el
rendimiento estaba muy por debajo del posible, y del deseado. Se realizaron los
diagramas de secuencia de hasta el sexto caso de uso, casi todos los
relacionados con la gestión de partida, intentando usar los valores que se
creían más adecuados. Así mismo, se empezó con la implementación del módulo de
gestión de usuarios. Estas funciones eran relativamente sencillas y era poco
probable que el trabajo hecho fuera a resultar incorrecto al tener la interfaz
definitiva.

El día 22 de diciembre tuvo lugar la reunión de jefes de grupo de la que surgió
la primera versión estable de la interfaz de comunicaciones, y que
posteriormente sólo sufriría cambios menores. En los días siguientes a esa
fecha, y con una idea mucho más clara de la arquitectura de la aplicación de la
que se tenía en octubre, se repasaron todos los diagramas realizados hasta la
fecha. Este hito dio por concluida la fase de elaboración.

\subsection{Fase de construcción}

\textit{Del 26 de diciembre al 27 de febrero}

El primer paso de esta fase fue reescribir el código para que se ajustara a los
cambios realizados en diseño, lo cual no supuso un gran esfuerzo.

Durante el mes de enero, el rendimiento fue también bajo debido a ser el
período de exámenes. En este mes se realizaron dos elementos importantes en
diseño y en implementación.

Por un lado, se empezó la implementación del caso de uso ``Conectarse a una
partida''. Este caso de uso es la base para todos los que vendrían a
continuación. En él se obtienen los datos de una partida, se configura dominio
creando el motor de juego, y se carga la interfaz de partida, la cual incluye
el mapa. Es una actividad bastante extensa que no estaría terminada hasta el 5
de febrero.

Paralelamente, los analista-diseñadores estuvieron trabajando en el caso de uso
``Atacar un territorio''. Este caso de uso está formado por una secuencia de
mensajes entre dos clientes. Estos mensajes son asíncronos, lo que implica
tener que guardar información sobre el ataque en curso. Son muchos pequeños
detalles que debían pensarse detenidamente para que la implementación
funcionara correctamente.

Durante el mes de febrero, los analista-diseñadores continuaron con su trabajo
hasta tener el diseño completo de la aplicación. A la vez, los programadores, y
en la recta final de esta fase también los \textit{testers}, implementaban o
probaban los diseños que iban siendo generados.

\subsection{Fase de transición}

\textit{Del 28 de febrero al 9 de marzo}

Para el 28 de febrero, y ya con la tranquilidad de disponer de otra semana más
para el desarrollo, la aplicación estaba casi concluida. Los casos de uso que
quedaran por implementar no supondrían un gran esfuerzo pues toda la
infraestructura software estaba ya creada.

Comenzaba el momento de realizar pruebas de integración con el servidor real en
producción. En esta pruebas se corrigieron errores de implementación en puntos
poco claros en la comunicación con el servidor, a la vez que se iban
corrigiendo errores en la interfaz, pues ésta no había sido probada con
\textit{JUnit}.

Tras una sesión de pruebas en conjunto con el equipo del servidor y otro equipo
cliente, no quedaba nada más por hacer.

\section{Riesgos}

\subsection{Riesgos externos}

El desarrollo de la aplicación ha sido afectado externamente por la tardía
especificación de la interfaz de comunicaciones con el servidor. Sin embargo,
este retraso fue sufrido en las primeras semanas del desarrollo, por lo que su
influencia ha sido relativamente pequeña, dando tiempo al equipo de recuperarse
del mismo.

\subsection{Riesgos técnicos}

Los mayores retrasos en el desarrollo se han sufrido debido a implementaciones
que no se ajustaban al diseño realizado. Errores de concepto que hacían que
otros casos de uso no funcionaran correctamente pues el comportamiento de las
clases no era el previsto en diseño.

Este continuo ir y venir entre diseñadores y programadores hizo que tareas que
se esperaran terminadas para un determinado plazo, tuvieran que ser aplazadas
por tener que volver a realizarse.

\subsection{Riesgos de planificación}

En los últimos días de febrero se disponía del diseño de varios casos de uso
que, en principio, podrían ser implementados en paralelo. No resultó ser el
caso, y debido a los retrasos que provocaba trabajar en un sólo caso de uso
explicados en el punto anterior, se decidió cambiar la forma de asignar tareas
a los programadores.

Hasta ese momento, un equipo de programadores realizaría un caso de uso
completo, creando código en todas las capas de la aplicación. Para que varias
personas pudieran trabajar a la vez, se agruparon las tareas de programación de
varios casos de uso. Con esto se consiguio una carga de trabajo mayor que
podría ser repartida más eficientemente entre varias personas.
