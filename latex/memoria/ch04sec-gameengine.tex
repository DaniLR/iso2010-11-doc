\section{Clase GameEngine}

{\small
\begin{tabular}{r|l}
Nombre del \textit{tester} & Jorge Colao Adán \\
& Daniel León Romero\\
Fecha de asignación & 1 de marzo de 2011 \\
Fecha de finalización & 7 de marzo de 2011 \\
Código bajo prueba & \texttt{GameEngine}
\end{tabular}
}

En este apartado se hablará sobre las pruebas realizadas con la herramienta \textit{JUnit} sobre esta clase.

La estrategia a seguir para las pruebas de estos apartados será \textit{Each Choice}. La elección de los valores de los parámetos, se realizarán con valores límite para los atributos númericos y valores propensos a error para el resto.

A la hora de realizar las pruebas en la clase GameEngine tenemos unos atributos privados a los que no podemos acceder. Para acceder a ellos se necesita la clase \textit{PrivateAccesor.java}, disponible en la página \url{http://onjava.com/pub/a/onjava/2003/11/12/reflection.html?page=2}. Por ejemplo, para acceder a la variable mCurrentAttack de la clase GameEngine, se tendría declarar un objeto de esta manera:
\begin{verbatim}
Object o = PrivateAccessor.getPrivateField(gameEngine, 
	"mCurrentAttack");
\end{verbatim}

Para poder probar los ataques primero tenemos que conectarnos a la partida. La conexión se realiza con el método connectToGame que tiene como argumentos el número de la partida a la que hay que conectarse y un objeto de la clase \textit{GameEventListener}. Para crear este objeto tenemos que crear una clase privada que implemente \textit{GameEventListener}.

\subsection{GameEngine::attackTerritory}

Este método tiene seis parámetros de entrada, por lo que tenemos que utilizar alguna estrategia de generación de casos de prueba. Hemos utilizado \textit{Each Choice}, combinando las entradas para que falle una y sólo una, en cada caso de prueba y poder saber en donde esta el error.

Para elegir tanto los valores de prueba como los casos de pruebas de los casos interesantes se ha analizado el comportamiento de este método (mediante caja blanca). Después de analizar el método nos hemos dado cuenta de que pude haber los siguientes errores:
\begin{itemize}
\item PendingAttackException
\item ArrayIndexOutOfBoundsException
\item IndexOutOfBoundsException
\item UnocupiedTerritoryException
\item NegativeValueException
\item NotEnoughUnitsException
\item InvalidTerritoryException
\end{itemize}
Esta comprobación es local, por ello si se lanza alguna excepción de las anteriores, no se realizará el ataque.

Estos son los valores de los atributos para los casos de prueba:
\begin{itemize}
\item \textbf{\texttt{src}}
\subitem Número incorrecto: \texttt{"\--1"}, \texttt{"41"} y \texttt{"42"}
\subitem Número correcto: \texttt{"0"}, territorio en el que se encuetra \textit{JorgeCA}

\item \textbf{\texttt{dst}}
\subitem Número incorrecto: \texttt{"\--1"}, \texttt{"0"}, \texttt{"41"} y \texttt{"42"}
\subitem Número correcto: \texttt{"2"}, territorio en el que se encuetra \textit{Aduran}

\item \textbf{\texttt{soldiers}}
\subitem Número incorrecto: \texttt{"\--1"} y \texttt{"21"}
\subitem Número correcto: \texttt{"0"} y \texttt{"20"}

\item \textbf{\texttt{cannons}}
\subitem Número incorrecto: \texttt{"\--1"} y \texttt{"7"}
\subitem Número correcto: \texttt{"0"} y \texttt{"6"}

\item \textbf{\texttt{missiles}}
\subitem Número incorrecto: \texttt{"\--1"} y \texttt{"2"}
\subitem Número correcto: \texttt{"0"} y \texttt{"1"}

\item \textbf{\texttt{icbm}}
\subitem Número incorrecto: \texttt{"\--1"} y \texttt{"7"}
\subitem Número correcto: \texttt{"0"} y \texttt{"6"}
\end{itemize}

La tabla siguiente recoge los resultados esperados.

{\footnotesize
\begin{longtable}[c]{lccc}
 & \textbf{Valores de Prueba} & \textbf{Objetivo del test} & \textbf{Resultado esperado} \\
\hline \hline
\endhead

Test1 & (0, 2, 0, 0, 1, 6)  & realizar ataque & Funcinamiento Correcto\\
Test2 & (0, 2, 0, 0, 1, 6)  & realizar 2 ataques & PendingAttack\\
Test3 & (-1, 2, 0, 0, 1, 6)  & src & ArrayIndexOutOfBounds\\
Test4 & (42, 2, 0, 0, 1, 6)  & src & IndexOutOfBounds\\
Test5 & (41, 2, 0, 0, 1, 6)  & src & UnocupiedTerritory\\
Test6 & (0, -1, 0, 0, 1, 6)  & dst  & ArrayIndexOutOfBounds\\
Test7 & (0, 42, 0, 0, 1, 6)  & dst & IndexOutOfBounds\\
Test8 & (0, 0, 0, 0, 1, 6)  & dst & InvalidTerritory\\
Test9 & (0, 2, -1, 0, 1, 6)  & soldiers & NegativeValue\\
Test10 & (0, 2, 21, 0, 1, 6)  & negTime & NotEnoughUnits\\
Test11 & (0, 2, 20, -1, 1, 6)  & cannons & NegativeValue\\
Test12 & (0, 2, 20, 7, 1, 6)  & cannons & NotEnoughUnits\\
Test13 & (0, 2, 20, 6, -1, 6)  & missiles & NegativeValue\\
Test14 & (0, 2, 20, 6, 2, 6)  & missiles & NotEnoughUnits\\
Test15 & (0, 2, 20, 6, 1, -1)  & icbm & NegativeValue\\
Test16 & (0, 2, 20, 6, 1, 7)  & icbm & NotEnoughUnits\\
Test17 & (0, 2, 0, 0, 0, 0)  & realizar ataque & Funcinamiento Correcto\\
Test18 & (0, 2, 20, 6, 0, 0)  & realizar ataque & Funcinamiento Correcto\\

\hline
\end{longtable}
}


\subsection{GameEngine::territoryUnderAttack}
\subsection{GameEngine::acceptAttack}
\subsection{GameEngine::requestNegotiation}
\subsection{GameEngine::resolveAttack}
\subsection{GameEngine::resolveNegotiation}
