\section{Crear una partida}

Guión de pruebas exploratorias para el método crear una partida:

\begin{enumerate}
\item Error la crear una partida con campos vacíos.
	\begin{itemize}
	\item Acciones previas: El servidor debe estar funcionando, haberse logeado con éxito y haber pulsado el botón ''Crear partida'' de la barra de herramientas.
	\item Acciones de prueba: En el dialogo de nueva partida se pulsa directamente en ''Crear Partida'', dejando en blanco los campos nombre, descripción y fecha, porque los demás estan por defecto a 60 segundos.
	\item Resultados esperados: No se crea la nueva partida y muesta el mensaje de error: ''¡Datos erróneos o insuficientes!''.
	\end{itemize}
	
\item Crear una partida, con descripción vacía.
	\begin{itemize}
	\item Acciones previas: El servidor debe estar funcionando, haberse logeado con éxito y haber pulsado el botón ''Crear partida'' de la barra de herramientas.
	\item Acciones de prueba: En el dialogo de nueva partida se escribe el nombre ''partida'' para el campo nombre de la partida y se añade la fecha actual. Los tiempos de juego se quedarán por defecto a 60 segundos. Se pulsa en ''Crear Partida'', dejando en blanco el campo descripción.
	\item Resultados esperados: Se crea una nueva partida, llamada "partida". En nuestro servidor de prueba no esta implementado la creación de una nueva partida, pero con el servidor del grupo de \textit{Jorge Barco} si funciona.
	\end{itemize}
	
\item Crear una partida, rellenando todos los campos.
	\begin{itemize}
	\item Acciones previas: El servidor debe estar funcionando, haberse logeado con éxito y haber pulsado el botón ''Crear partida'' de la barra de herramientas.
	\item Acciones de prueba: En el dialogo de nueva partida se escribe el nombre ''partida'' para el campo nombre de la partida, como descripción se ha puesto ''partida guera mundo'' y se añade la fecha actual. Los tiempos de juego se quedarán por defecto a 60 segundos. Se pulsa en ''Crear Partida''.
	\item Resultados esperados: Se crea una nueva partida, llamada "partida" y con la descripción ''partida guera mundo''.
	\end{itemize}
\end{enumerate}

El problema de las excepciones de rango de los camos de tiempo, no se puede producir, porque en la interzar se evita que se puedan poner números menores de 1.

\subsection{Feedback}

Al realizar la pruebas exploratorias de este método nos dimos cuenta que se podian crear partidas con campos indispensables vacíos. Por lo que se informó al desarrollador pertinente.
