\section{Clase UserManager}

\subsection{UserManager::registerUser}

{\small
\begin{tabular}{r|l}
Nombre del \textit{tester} & Antonio Gómez Poblete \\
Fecha de asignación & 27 de enero de 2011 \\
Fecha de finalización & 30 de enero de 2011 \\
Código bajo prueba & \texttt{UserManager::registerUser}
\end{tabular}
}

A continuación se detallarán las pruebas de desarrollo (Pruebas unitarias con \textit{Junit}).

Lista de los valores de prueba para cada atributo.
El criterio elegido para todos los valores de prueba (test data) ha sido: Añadir valores interesantes propensos a error (Conjetura de error).

\begin{itemize}
\item \textbf{\texttt{login}}
\subitem \textit{null}
\subitem Cadena vacía
\subitem Usuario existente: \texttt{"JorgeCA"}
\subitem Usuario no existente: \texttt{"LuisAn"}

\item \textbf{\texttt{passwd}}
\subitem \textit{null}
\subitem Cadena vacía
\subitem Cadenas no vacías: \texttt{"jorge"}, \texttt{"luis"}

\item \textbf{\texttt{email}}
\subitem \textit{null}
\subitem Cadena vacía
\subitem Valores incorrectos: \texttt{"jorge"}, \texttt{"jorge@"}, \texttt{"jorge@gmail"}, \texttt{"jorge@gmail."}
\subitem Valores correctos: \texttt{"jorge.colao@gmail.com"}, \texttt{"luis@gmail.com"}
\end{itemize}

La estrategia para obtener los casos de prueba elegida ha sido
\textit{each choice} (Test 1 a 3 y 8 a 12 ) y añadir algunos casos
interesantes (Test 4 a 7).
El test 13 se ha elaborado para ver el comportamiento de registerUser cuando el servidor se ha desconectado.

Para elegir tanto los valores de prueba como los casos de pruebas se ha analizado el comportamiento de este caso de uso (caja blanca). Tras este análisis se ha llegado a la conclusión de que hay en dos lugares en los que se pueden encontrar errores: Si al método registerUser se le da un parámetro erróneo (throw EmptyStringException, MalformedEmailException, NullPointerException), o si el nombre del usuario ya existe en el servidor (throw UserAlreadyExistsException). La primera comprobación es local, por ello si se lanza la excepcion InvalidArgumentException no se hará nada en el servidor y no se analizará si el usuario ya existe o no.

Además, si observamos las comprobaciones que se hacen en el cliente se ve que si login == null, independientemente del valor de los demás atributos se lanzará  NullPointerException, por esa razón es independiente el valor de los demas atributos, lo que hace que muchos casos de prueba no sean necesarios de probar. Por otro lado, para probar qué pasa si el email es \textit{null} el resto de los parámetros deberán ser correctos, lo que añade algunos casos interesantes. Además de con \textit{null} lo comentado en este parrafo sucede con el resto de los parámetros.


Teniendo en cuenta las consideraciones anteriormente mencionadas, la tabla completa de los casos de prueba y los resultados esperados son:

{\footnotesize
\begin{longtable}[c]{lccc}
 & \textbf{Valores de Prueba} & \textbf{Objetivo del test} & \textbf{Resultado esperado} \\
\hline \hline
\endhead

T1 & (vacío, vacío, vacío)  & login & EmptyStringException\\
T2 & (null, null, null) & login & NullPointerException\\
T3 & (JorgeCA, jorge, jorge.colao@gmail.com) & Usuario ya existente & UserAlreadyExistst\\
T4 & (JorgeCA, vacío, jorge.colao@gmail.com) & passwd   & EmptyStringException\\
T5 & (JorgeCA, null, jorge.colao@gmail.com) & passwd   & NullPointerException\\
T6 & (JorgeCA, jorge, vacío) & email   & EmptyStringException\\
T7 & (JorgeCA, jorge, null) & email   & NullPointerException\\
T8 & (JorgeCA, jorge, jorge) & email & MalformedEmailException\\
T9 & (JorgeCA, jorge, jorge@) & email & MalformedEmailException\\
T10 & (JorgeCA, jorge, jorge@gmail) & email & MalformedEmailException\\
T11& (JorgeCA, jorge, jorge@gmail.) & email  & MalformedEmailException\\
T12& (LuisAn, luis, luis@gmail.com) & email  &  El usuario \\
& &  & queda registrado\\
T13& (Angel\&Duran, angel, a@d.com)  & Comportamiento registerUser & RemoteException \\
\hline
\end{longtable}
}

\subsection{UserManager::login}

{\small
\begin{tabular}{r|l}
Nombre del \textit{tester} & Angel Durán Izquierdo \\
Fecha de asignación & 27 de enero de 2011 \\
Fecha de finalización & 2 de febrero de 2011 \\
Código bajo prueba & \texttt{UserManager::createSession}
\end{tabular}
}

A continuación se detallarán las pruebas de desarrollo (Pruebas unitarias con \textit{JUnit}).

Lista de los valores de prueba para cada atributo.
El criterio elegido para todos los valores de prueba (test data) ha sido: Añadir valores interesantes propensos a error (Conjetura de error).

\begin{itemize}
\item \textbf{\texttt{login}}
\subitem \textit{null}
\subitem Cadena vacía
\subitem Usuario existente: \texttt{Angel}
\subitem Usuario no existente: \texttt{ADuran}
\subitem Nombre de usuario registrado pero introducido con mayusculas: \texttt{ADuran}
\subitem Usuario existente con carácteres especiales: \texttt {Angel\&Duran}
\subitem Usuario existente con numeros: \texttt{1111, -1}

\item \textbf{\texttt{passwd}}
\subitem \textit{null}
\subitem Cadena vacía
\subitem Cadenas no vacías: \texttt{angel}
\subitem Cadenas con carácteres especiales: \texttt{22\&22}
\subitem Cadenas con numeros: \texttt{2222}

\end{itemize}

La estrategia para obtener los casos de prueba elegida ha sido
\textit{each choice} eligiendo los casos mas interesantes.

Para elegir tanto los valores de prueba como los casos de pruebas se ha analizado el comportamiento de este caso de uso (caja blanca). Tras este análisis se ha llegado a la conclusión de que hay en dos lugares en los que se pueden encontrar errores: Si al método createSession se le da un parámetro erróneo (throw InvalidArgumentException), o si el nombre del usuario no se ha registrado en el servidor (throw UserAlreadyExistsException).

Por otro lado se comprueba si al crear una sesion cuando ya se existía una, la anterior se cierra correctamente.


Teniendo en cuenta las consideraciones anteriormente mencionadas, la tabla completa de los casos de prueba y los resultados esperados son:

{\footnotesize
\begin{longtable}[c]{lccc}
 & \textbf{Valores de Prueba} & \textbf{Objetivo del test} & \textbf{Resultado esperado} \\
\hline \hline
\endhead

Test1 & (vacío, vacío)  & login-pass & InvalidArgument\\
Test2 & (Aduran, vacío) & pass & InvalidArgument\\
Test3 & (vacío, angel) & login & UserAlreadyExistst\\
Test4 & (Aduran, angel) & login-pass   & Sesión creada correctamente\\
Test5 & (null, angel) & login   & InvalidArgument\\
Test6 & (Aduran, null) & pass   & InvalidArgument\\
Test7 & (null, null) & login-pass   & InvalidArgument\\
Test8 & (ADuran, angel) & login & WrongLoginException\\
Test9 & (Aduran, Angel) & pass & WrongLoginException\\
Test10 & (1111, 22\&22) & login-pass & Sesión creada correctamente\\
Test11& (-1, 2222) & login-pass  & Sesión creada correctamente\\
Test12& (Angel\&Duran, angel) & login  &  Sesión creada correctamente \\
Test13 & (Aduran y ricki, angel y ricki) & Comportamiento createSession & Sesión creada correctamente\\
Test14 & (Aduran, angel) & Comportamiento createSession & RemoteException\\
Test15 & (Aduran, angel) & Comportamiento createSession & Sesión cerrada correctamente\\
Test16 & (Aduran, angel) & Comportamiento createSession & Sesión creada correctamente\\

\hline
\end{longtable}
}
